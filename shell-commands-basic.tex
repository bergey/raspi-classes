%% © Daniel Bergey 2015
%% CC-BY

\documentclass{article}
\pagestyle{empty}		% no page numbers, thanks.
\usepackage[top=0.7in,bottom=0.7in,left=0.7in,right=0.7in]{geometry}

\usepackage{setspace}
\setstretch{1.1}
\parskip=0.15em

%% \usepackage{fontspec}

\usepackage{tabularx}
\setlength{\extrarowheight}{2.3\baselineskip}
\usepackage{ragged2e}
\newcolumntype{Y}{>{\RaggedRight\arraybackslash}X}
\usepackage{booktabs}

\begin{document}
%% \setmainfont{TeX Gyre Pagella}

\section{Common Linux Shell Commands}

\begin{tabularx}{\textwidth}{@{} Y Y Y @{}}
  Command & Description & Notes \\
  \midrule
   ssh USER@ADDRESS \newline ssh fluttershy@raspi0 & login to another computer & \\
   ls & list files \& folders in current folder & \\
   edit FILE \newline edit hello.py & open FILE in the editor on your Mac & \\
   python & start Python in interactive mode & \\
   python PROGRAM \newline python hello.py & run the python PROGRAM & \\
   cd DIR \newline cd inventwithpython & change directory (folder) to DIR & \\
   cd & go back to your  HOME directory & \\
   rm FILE \newline rm whoops.txt & remove (delete) FILE.  \newline This is permanent! & \\
   mv FROM TO \newline mv hello.py last-week/  & rename a file or move it to another folder & \\
   pwd & print working directory \newline show the full name of the current folder & \\
   mkdir & make a new directory & \\
   nano FILE \newline nano hello.py & open FILE in an editor in Terminal \newline Can be helpful if edit isn't set up correctly. & \\
\end{tabularx}
 
\end{document}
